% !TeX root = ../main.tex

\chapter{系统测试和分析}

软件测试是软件开发的最后一道程序,软件测试分为功能性测试和非功能性
测试。功能性测试是为了保证需求分析的预期功能都被实现,给用户提供满意的
交付。非功能性测试是为了保证软件的性能及使其实现可持续发展,主要关注于
系统的可维护性、扩展性、健壮性、兼容性等。此外,对于安全性要求较高软件
系统,还需要进行大量的安全性测试,来保证安全性。本文的上一章节介绍了本
中文时间表达式信息抽取系统的详细设计及实现,本章将对实现的中文时间表达式信息抽取系统进行测试,其主
要目的是验证本中文时间表达式信息抽取系统是否能够准确有效的支持中文时间表达式的识别解析, 是否能正确的存储到数据库中, 是否能正确完成用户界面的渲染。

\section{测试方案}

中文是除英文外适用人群最多的语言,也不乏含有中文时间表达式的语料,比较具有权威
性的是国际上的任务赛事中的语料,例如 TempEval-2,ACE2007,ACE2005,TERN2004,其
中部分语料是非公开。这些语料只含有识别任务(任务 R)或归一化任务(任务 N)。国内常
用的训练语料是采用 Timex2 标注的 1980 年人民日报,也只有识别任务。现在拟标注部分具
有 N 任务的语料,并将语料随机插入 R 任务语料中,替代 R 任务中的时间表达式,形成 R +
N 混合语料,并将 R + N 语料按 6:2:2 的比例划分为训练集、开发集与测试集,用 Fβ-measure
评价系统在 R + N 语料上的表现。

系统采用单元测试的方式对各个模块的功能函数进行正确性测试,在提供的
测试环境中对系统的各项功能进行功能性测试,对系统整体的性能做性能测试,
最后对测试结果进行详细的分析和总结。主要的测试方案如下:
(1)web 客户端:通过浏览器访问本系统,在 web 页面上的组件上发送各类查询需求,检查返回数据是否完整,页面渲染是否符合预期结果。
(2)服务端:对 web 客户端的查询请求进行解析和检查,检查查询请求是
否都能正确到达服务端,并且被服务端所处理。根据服务端的资源消耗分析系统
的性能瓶颈。
(3)数据库:通过对数据库接口进行测试,检查数据库接口对于指定查询
需求是否能返回数据库中存在的正确数据,数据库对查询请求的响应时间是否具
有可用性。
(4)客户端、服务端以及数据库访问接口模块中的代码做单元测试。

\section{测试平台}



