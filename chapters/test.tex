% !TeX root = ../main.tex

\chapter{系统测试和分析}ss

软件测试是软件开发的最后一道程序,软件测试分为功能性测试和非功能性
测试。功能性测试是为了保证需求分析的预期功能都被实现,给用户提供满意的
交付。非功能性测试是为了保证软件的性能及使其实现可持续发展,主要关注于
系统的可维护性、扩展性、健壮性、兼容性等。此外,对于安全性要求较高软件
系统,还需要进行大量的安全性测试,来保证安全性。本文的上一章节介绍了本
中文时间表达式信息抽取系统的详细设计及实现,本章将对实现的中文时间表达式信息抽取系统进行测试,其主
要目的是验证本中文时间表达式信息抽取系统是否能够准确有效的支持中文时间表达式的识别解析, 是否能正确的存储到数据库中, 是否能正确完成用户界面的渲染。

\section{测试方案}

中文是除英文外适用人群最多的语言,也不乏含有中文时间表达式的语料,比较具有权威
性的是国际上的任务赛事中的语料,例如 TempEval-2,ACE2007,ACE2005,TERN2004,其
中部分语料是非公开。这些语料只含有识别任务(任务 R)或归一化任务(任务 N)。国内常
用的训练语料是采用 Timex2 标注的 1980 年人民日报,也只有识别任务。现在拟标注部分具
有 N 任务的语料,并将语料随机插入 R 任务语料中,替代 R 任务中的时间表达式,形成 R +
N 混合语料,并将 R + N 语料按 6:2:2 的比例划分为训练集、开发集与测试集,用 Fβ-measure
评价系统在 R + N 语料上的表现。

系统采用单元测试的方式对各个模块的功能函数进行正确性测试,在提供的
测试环境中对系统的各项功能进行功能性测试,对系统整体的性能做性能测试,
最后对测试结果进行详细的分析和总结。主要的测试方案如下:
(1)web 客户端:通过浏览器访问本系统,在 web 页面上的组件上发送各类查询需求,检查返回数据是否完整,页面渲染是否符合预期结果。
(2)服务端:对 web 客户端的查询请求进行解析和检查,检查查询请求是
否都能正确到达服务端,并且被服务端所处理。根据服务端的资源消耗分析系统
的性能瓶颈。
(3)数据库:通过对数据库接口进行测试,检查数据库接口对于指定查询
需求是否能返回数据库中存在的正确数据,数据库对查询请求的响应时间是否具
有可用性。
(4)客户端、服务端以及数据库访问接口模块中的代码做单元测试。

\section{测试环境}

本代码检查工具可以部署在 Linux 和 Windows 系统环境中。在实际的应用
中,实时检查功能一般是运行在 Windows 环境中,集成到具有图形界面的 IDE
中。而批量检查功能运行时间长,且消耗资源,一般运行在 Linux 服务器中。表
6.1 为本代码检查工具在 Window 环境下运行实时检查功能所需要的配置。

\begin{table}[h]
    \centering
    \caption{windows软件环境}
    \begin{tabular}{|*{3}{c|}}
        \hline
        项目 & 软件 & 版本 \\
        \hline
        操作系统 & Windows 10 & 1909 \\
        \hline
        编辑器 & Visual Studio Code & 1.42 \\
        编程语言 & Python & 3.6.9 \\
        \hline
        代理服务器 & Nginx & 1.17.7 \\
        \hline
        工具库 & SciKit-learn &  0.22 \\
        \hline 
    \end{tabular}
\end{table}

\begin{table}[h]
    \centering
    \caption{Linux软件环境}
    \begin{tabular}{|*{3}{c|}}
        \hline
        项目 & 软件 & 版本 \\
        \hline
        操作系统 & Ubuntu  & 18.04 LTS \\
        \hline
        编辑器 & Visual Studio Code & 1.41 \\
        编程语言 & Python & 3.7.2 \\
        \hline
        代理服务器 & Nginx & 1.17.2 \\
        \hline
        工具库 & SciKit-learn &  0.21.1 \\
        \hline 
    \end{tabular}
\end{table}

\section{功能性测试}

\subsection{交互功能测试}

\section{非功能性测试}

\subsection{易用性测试}

为了满足易用性测试,本系统在提供给用户使用时用户不需要关心数据库以
及前端页面的具体实现,用户只需根据自己的数据分析需求编写指定的
SQL 语句,前端页面会自动渲染数据查询结果,以可视化的形式呈现数
据查询结果。因此发布的以太坊数据高速抽取和分析系统满足易用性的需求。

\subsection{兼容性测试}

为了满足兼容性测试,本系统的前端 Grafana 页面分别在谷歌浏览器、Edge
浏览器、火狐浏览器、360 浏览器进行了测试,以上浏览器均能在 Grafana 页面
上呈现功能测试中的各项数据,因此以太坊数据高速抽取和分析系统对以上各种
浏览器兼容。

\subsection{可维护性测试}

软件系统的可维护性可以用模块之间的耦合度、单元测试和集成测试的覆盖
率、代码规范等指标度量。本系统在开发过程中采用面向对象的设计方法,严格
遵循单一责任、低耦合高内聚等原则来进行系统模块的设计,减少模块与模块之
间的耦合度、增加模块内部的功能复用。系统在开发过程中也会在关键代码处添
加日志,当异常和错误情况发生时,开发人员能够快速通过日志进行问题解决。
综合以上分析,以太坊数据高速抽取和分析系统满足可维护性需求。

