% !TeX root = ../main.tex

\chapter{总结与展望}

本章主要用来总结论文所做工作,在项目中的创新点和技术难点,并给出未来工作的展望和应该补足的一些工作。

本中文时间表达式识别与解析算法还具有一定的创新性,基于概率上下文无关语法,并应用全新的时间表达式的组合语义标注,以此来构建中文时间信息抽取系统;
此外,基于概率上下文无关语法,摆脱了完全基于规则,即使用正则表达式识别文本中的时间时,撰写规则的繁琐程度,同时用统计学的方法解决时间表达式识别中存在的冲突问题,识别结果更具有鲁棒性;
最后,应用时间表达式的组合语义标注,改良了 ISO-TimeML Timex3 标注格式的局限性,拓宽了时间表达式归一化后的语义表示范围。
除了核心算法的设计之外,本项目还将算法作为工具的一部分,完成了与后端交互的逻辑,即将算法组件作为外观模式层,完成与服务层的交互,同时为系统设计了一套用户交互界面和开发人用可应用编程接口,
方便用户使用和开发人员二次开发。 该项目已经能初步交付使用。

项目结束后,仍然不能算一个完全品,究其原因,在于基于概率无关的上下文语法仍然有一些问题无法解决:
\begin{enumerate}
    \item[(1)] 目前的研究中尚无将中文时间表达式作为语法来识别,现采用的概率上下文无关语法
    虽然能简化规则的撰写,但大量的语法规则的编写还是需要手工完成;
    \item[(2)] 语料库的缺乏会降低时间表达式归一化过程的准确度(Precision),主要体现为同一时
    间表达式解析出的多棵语法树间概率分布与真实数据分布不相匹配,当前还需要解决语料缺
    乏的问题;
    \item[(3)] 中文时间表达式归一化过程采用的 SCATE 结构与英语语境下的时间表达式相差比较
    大,需要重新设计以契合中文语境,此外,最终获得根结点上的嵌套表达式的计算方式,也需
    要重新设计。
\end{enumerate}

自笔者初步完成论文草稿时,自然语言处理中的时间表达式识别与解析的进展只能算小有进步。 
中文时间表达式识别与解析影响较大的工作还时停留在2017年的论文中。 有关中文时间表达式识别与解析的研究道阻且长。



