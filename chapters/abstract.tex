% !TeX root = ../main.tex

\ustcsetup{
  keywords = {
      信息抽取, 中文时间表达式, 统计句法分析
    },
  keywords* = {
      Information Extraction, Chinese Temporal Expression, Statistical parsing
    },
}

\begin{abstract}

  自然语言处理领域中,时间和日期作为一类重要的命名实体和语义载体而存在。
  时间信息抽取在问答系统、多轮对话、信息检索等任务中有着广泛的应用。
  现有的时间信息抽取的解决方案中,因为中文本身的复杂性和采用标注格式的局限性,
  最终产生的结构化的时间数据不尽如人意,不能较好的表现时间实体的内在语义。
  本文主要基于概率上下文无关语法解决中文时间信息抽取中的这一系列问题。
  在此基础上,论文旨在设计并实现一个完整的中文时间信息抽取系统,以提升时间处理模块作为其他自然语言处理任务组件的使用体验。

  本文首先对中文时间信息抽取的一系列痛点问题做出了分析, 将中文时间信息抽取分解为识别和解析两个具体需求, 并基于文档语料的特点的采用MongoDB存储数据。
  其次, 采用分层架构的思想将系统分为四层架构, 并依据四层架构的特点设计出功能明晰的不同模块。
  然后, 通过上下文无关语法的形式简化传统方法中手工驱动的编写时间表达式规则的繁琐过程,并采用统计学习方法解决识别出的时间信息可能存在的多种语义理解方式的歧义性冲突。
  对于识别出的时间信息,我们再利用时间粒度天然具有层次性和顺序性的语义特征,构建时间信息的语义表示,以改善主流方法中归一化后的时间信息抽取格式的局限性。

  最后, 本文对中文时间信息抽取系统进行测试,测试结果表明本中文时间信息抽取系统均已满足预期要求。
  本中文时间信息抽取系统现已投入使用,响应度以及压力测试也能够很好的证明其能满足一定流量的实时查询需求,为企业用户提供良好的时间信息抽取服务。

\end{abstract}

\begin{abstract*}
  In the field of natural language processing, time and date exist as a kind of important named entities and semantic carriers.
  Temporal information extraction is widely used in question answering system, multi-round dialogue, information retrieval and other tasks.
  This paper is mainly based on probabilistic context-free grammar to solve this series of problems in Chinese temporal information extraction.

  This paper first analyzes a series of problems of Chinese temporal information extraction, and decomposes Chinese temporal information extraction into two specific requirements for recognition and resolution, 
  and uses MongoDB to store data based on the characteristics of the document corpus.
  Secondly, the system is divided into a four-tier architecture using the idea of ​​a layered architecture, and different modules with clear functions are designed according to the characteristics of the four-tier architecture.
  Then, the cumbersome process of writing temporal expression rules manually driven in traditional methods is simplified through the form of context-free grammar, 
  and statistical learning methods are used to resolve the ambiguity conflicts of multiple semantic understanding methods that may exist in the identified temporal information.
  For the identified temporal information, we re-use the semantic features of time granularity that are naturally hierarchical and sequential to construct 
  a semantic representation of temporal information to improve the limitations of the normalized temporal information extraction format in mainstream methods.

  Finally, this paper tests the Chinese temporal information extraction system. 
  The test results show that the Chinese temporal information extraction system has met the expected requirements.
  This Chinese temporal information extraction system has been put into use, 
  and the responsiveness and stress test can also prove that it can meet the real-time query requirements of a certain flow, and provide good temporal information extraction services for enterprise users.


\end{abstract*}
