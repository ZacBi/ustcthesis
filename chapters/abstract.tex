% !TeX root = ../main.tex

\ustcsetup{
  keywords = {
      信息抽取, 中文时间表达式, 统计句法分析
    },
  keywords* = {
      Information Extraction, Chinese Temporal Expression, Statistical parsing
    },
}

\begin{abstract}
  自然语言处理领域中,时间和日期作为一类重要的命名实体和语义载体而存在。
  时间信息抽取在问答系统,多轮对话,信息检索等任务中有着广泛的应用。
  流行的时间信息抽取的解决方案中,因为中文本身的复杂性,和采用标注格式的局限性,最终产生的结构化的时间数据不尽如人意,不能较好的表现时间实体的内在语义。

  本文主要研究基于概率上下文无关语法和全新的时间表达式语义组合标注解决中文时间信息抽取中的一系列问题。
  通过上下文无关语法的形式简化传统方法中手工驱动的编写时间表达式规则的繁琐过程,并采用统计学习方法解决识别出的时间信息可能存在的多种语义理解方式的歧义性冲突。
  对于识别出的时间信息,我们再利用时间粒度天然具有层次性和顺序性的语义特征,构建时间信息的语义表示,以改善主流方法中归一化后的时间信息抽取格式的局限性。

  在此基础上,论文旨在设计并实现一个完整的中文时间信息抽取系统,以提升时间处理模块作为其他自然语言处理任务组件的使用体验,同时在某些评测任务中达到或接近顶尖水平,并探索中文时间信息抽取研究的新方向。
\end{abstract}

\begin{abstract*}
  In the field of natural language processing, time and date exist as a kind of important named entities and semantic carriers.
  temporal information extraction is widely used in question answering system, multi round dialogue, information retrieval and other tasks.
  In the popular temporal information extraction solutions, due to the complexity of Chinese itself and the limitations of annotation format, the final structured time data is not satisfactory and can not better express the internal semantics of time entities.

  This paper mainly studies a series of problems in Chinese temporal information extraction based on probabilistic context free grammar and a new semantic combination annotation of time expression.
  Through the form of context free grammar, the cumbersome process of manually driven writing time expression rules in traditional methods is simplified, and the statistical learning method is used to solve the possible ambiguity conflict of multiple semantic understanding methods of the identified time information.
  For the identified temporal information, we use the natural hierarchical and sequential semantic features of time granularity to construct the semantic representation of time information, so as to improve the limitations of the normalized time information extraction format in the mainstream methods.

  On this basis, this paper aims to design and implement a complete Chinese temporal information extraction system to improve the use experience of time processing module as a component of other natural language processing tasks, reach or approach the top level in some evaluation tasks, and explore a new direction of Chinese time information extraction research.
\end{abstract*}
